\chapter{Register automata}
\label{sec:register-automata}

A data word is a word where each letter carries two pieces of information: a \emph{label} from a finite set, and a \emph{data value} from an infinite set. Here is a picture:
\mypic{107}
For the rest of Part I, fix a countably infinite set $\atoms$. Elements of this set, called the \emph{atoms}, will be used for the data values. 
Formally, a \emph{data word} over a finite set of labels $\Sigma$ is defined to be a word in
\begin{align*}
	w \quad \in \quad \big(\underbrace{\Sigma}_{\text{label}} \quad \times \underbrace{\atoms}_{\text{data value}}\big)^*.
\end{align*}
When describing properties of data words, we will be able to test the labels explicitly by asking questions like 
\begin{center}
	does the second letter have $a \in \Sigma$ as its label?
\end{center}
but we will only test the data values for equality, e.g. ask 
\begin{center}
	do the third and fifth letters have the same data value?
\end{center}
Later in the book, we will formalise what it means to only test data values for equality, but for now the intuitive understanding should be enough. 

\begin{myexample}\label{ex:data-languages}
By abuse of notation, we assume that a word in $\atoms^*$ is also a data word, which uses no labels. Here are examples of languages of data words, in all of these examples we use no labels:
\begin{enumerate}
	\item the first data value is the same as the last data value;
\item some data value appears twice;
\item no data value appears twice;
\item the first data value appears again;
\item every three consecutive data values are pairwise distinct.
\end{enumerate}	
\end{myexample} 

We will introduce automata models for data words that recognise the above languages. These models use registers to talk about data values.



\section{Nondeterministic register automata}
We begin our discussion with some of the simplest automaton models for data words, namely nondeterministic and deterministic register automata\footnote{Register automata where introduced in~\cite{DBLP:journals/tcs/KaminskiF94}, under the name of \emph{finite memory automata}, together with a decidability proof for the emptiness problems in the deterministic and nondeterministic one-way cases (Theorem~\ref{thm:register-decidable-emptiness} in this text). The presentation using syntactic and semantics equivariance, in particular Lemma~\ref{lem:two-types-of-equivariance}, is essentially due to~\cite{DBLP:journals/mst/Bojanczyk13,DBLP:journals/corr/BojanczykKL14}.
 }. 
\begin{definition}[Nondeterministic register automaton]
	The syntax of a \emph{nondeterministic register automaton} consists of:
\begin{itemize}
	\item a finite set $\Sigma$ of \emph{labels};
	\item a finite set $\locations$ of \emph{locations}\footnote{We use the name location instead of state because the state of the automaton will store additional information, namely the contents of the registers.};
	\item a finite set $R$ of \emph{register names};
	\item an \emph{initial location} $\ell_0 \in \locations$ and a set of \emph{accepting locations} $F \subseteq \locations$;
	\item a \emph{transition relation}
\begin{align}\label{eq:register-automaton-transitions}
 \delta \subseteq \underbrace{\locations \times \overbrace{(\atoms \cup \set \bot)^R}}^{ \qquad \substack{\text{register valuations, i.e.}\\\text{partial functions}\\ \text{from registers}\\ \text{to atoms}}
 }
 _{\text{states}} \times \underbrace{\Sigma \times \atoms}_{\text{input}} \times \underbrace{\locations \times (\atoms \cup \set \bot)^R}_{\text{states}}
\end{align}
 subject to an equivariance condition described below.	
\end{itemize}
\end{definition}



The automaton is used to accept or reject data words with labels $\Sigma$, i.e.~words where each position is labelled by $\Sigma \times \atoms$. After processing part of the input, the automaton keeps track of a \emph{state}, which is defined to be a location plus a register valuation. Initially, the state consists of the initial location and a completely undefined register valuation. For each input letter, the state is updated according to the transition relation $\delta$, and the automaton accepts if at the end of the input word the state is accepting, in the sense that the location belongs to the accepting set.

How is the transition relation described? Since the state space is infinite, some restrictions on the transition relation are needed to represent it in a finite way. We choose the following restriction, called \emph{equivariance}: the transition relation can only compare atoms with respect to equality, and is not allowed to depend on any specific atoms. Equivariance can be formalised in two different ways, as described below.

\paragraph{Semantic equivariance.} A permutation $\pi : \atoms \to \atoms$ of the atoms (i.e.~a bijection from the atoms to themselves) can be applied to states in the natural way, and therefore also to triples in the transition relation $\delta$ (the locations and undefined values are not affected, only the atoms). Here is a picture:
\mypic{108}
We say that $\delta$ is \emph{semantically equivariant} if the set of transitions is invariant under actions of atom permutations, i.e.
\begin{align*}
 \pi(t)\in \delta \qquad \mbox{for every $t \in \delta$ and every permutation $\pi : \atoms \to \atoms$}.
\end{align*}
The advantage of semantic equivariance is that the definition is short, and easy to generalise to other models like alternating automata or pushdown automata. The disadvantage is that it is not clear how to represent a semantically equivariant transition relation, e.g.~for the input of a nonemptiness algorithm. The converse situation holds for syntactic equivariance, as presented below.
\paragraph{Syntactic equivariance.} We say that $\delta$ is \emph{syntactically equivariant} if it can be defined by a finite boolean combination of constraints of the following types:
\begin{enumerate}
	\item the location in the source / target state is $\ell \in \locations$;
	\item the label in the input letter is $a \in \Sigma$;
	\item register $r \in R$ is undefined in the source / target state;
	\item the atom in the input letter is the same as in register $r \in R$ of the source / target state;
	\item \label{it:sem-equiv-registers-equal} register $r \in R$ of the source / target state stores the same atom as register $s \in R$ of the source / target state.
\end{enumerate}
In the above, source / target means that the constraint can be instantiated with either ``source'' or ``target''. For example, 
in item~\ref{it:sem-equiv-registers-equal}, there are four possibilities regarding the choice of source vs target, since the choice is taken independently for $r$ and $s$. 


\begin{lemma}\label{lem:two-types-of-equivariance}
Semantic and syntactic equivariance are the same.
\end{lemma}
\begin{proof}
	It is not difficult to see that semantically equivariant subsets of the set~\eqref{eq:register-automaton-transitions} are closed under boolean combinations. Since the bijections of data values do not affect satisfaction of the constraints 1-5 used in the definition of syntactic equivariance, it follows that syntactic equivariance implies semantic equivariance.
	
	We now show that semantic implies syntactic. Define an \emph{orbit of transitions} to be a set of the form 
	\begin{align*}
		\set{\pi(t) : \text{$\pi$ is a permutation of the atoms}} \qquad \text{for some transition $t \in \delta$}.
	\end{align*}
	Here is a picture of an orbit of transitions:
	\mypic{109}
	 nonempty subset of the set~\eqref{eq:register-automaton-transitions} which is semantically equivariant and which is minimal for that property with respect to inclusion. 
	\begin{claim}
		Every orbit of transitions is syntactically equivariant.
	\end{claim}
	\begin{proof}[Proof of the claim]		
	An orbit of transitions is uniquely defined by its locations, which registers are undefined, what is the label of the input letter, and what is the equality type of the tuple of atoms in the defined registers and the input letter. All of this information can be expressed using the constraints 1-5 in the definition of syntactic equivariance.
	\end{proof}
	
	Once the number of registers and locations is fixed, there are finitely many possible constraints as in the definition of syntactic equivariance. Boolean combinations make the number of possibilities grow, but it remains finite. Therefore, thanks to the above claim, there are finitely many possible orbits of transitions. Finally, every semantically equivariant relation is easily seen to be the union of the orbits contained in it. This union is finite, and each part of the union is syntactically equivariant, hence the result follows.
\end{proof}


This completes the definition of nondeterministic register automata: the transition relation is required to be equivariant in either of the two equivalent senses defined above. The transition relation is called \emph{deterministic} if the source state and the input letter determine uniquely the target state.


\begin{myexample}
Here is a deterministic register automaton which recognises language 1 from Example~\ref{ex:data-languages}, i.e.~the words in $\atoms^*$ where the first and last data values are equal. The automaton stores the first data value in its register, and then toggles between accepting or rejecting states depending on whether the input agrees with the register. Here is a picture: \mypic{4}
The above picture should be interpreted as follows. There are three locations, standing for the three coloured circles, with initial and final locations depicted using dangling arrows. Since there is one register, a state consists of a location and an atom, with the atom possibly undefined. Such states can be found in the picture above. For every pair of distinct atoms $a \neq b$, we add a transition from the above picture to the automaton. Note how every arrow in the picture corresponds to an orbit of transitions.

The method of drawing above has its limitations. For example, if we wanted to add a transition that would involve the orange location with an undefined register, we would need to draw a separate instance of the orange state. 
\end{myexample}


\begin{myexample}\label{ex:no-complement}
	Languages recognised by nondeterministic register automata are not closed under complement. Consider the language 
	\begin{align*}
		L = \set{w \in \atoms^* : \text{some data value appears twice in $w$}}.
	\end{align*} 
	This language is recognised by a nondeterministic register automaton with one register and three locations. The automaton uses nondeterminism to guess the repeating data value. Here is a picture:
	\mypic{122}
	We now show that the complement of this language -- namely the words where no data value repeats -- is not recognised by any nondeterministic register automaton. Toward a contradiction, suppose that there is such a nondeterministic automaton, say with $<k$ registers, and consider an accepting run over a word with $2k$ distinct data values:
	\begin{align}\label{eq:alleged-accepting-run}
	q_0 \stackrel {a_1} \to q_1 \stackrel {a_2} \to \cdots \stackrel {a_{2k}} \to q_{2k}
	\end{align}
	Since the automaton has $<k$ registers then there must be some atoms
	\begin{align*}
		a \in \set{a_1,\ldots,a_k} \qquad b \in \set{a_{k+1},\ldots,a_{2k}}
	\end{align*}
	such that neither $a$ nor $b$ appear in the registers of state $q_k$. Let $\pi$ be the atom permutation which swaps $a$ and $b$. If we apply $\pi$ to the second half of the run in~\eqref{eq:alleged-accepting-run}, then we also get an accepting run (because transitions and accepting states are closed under applying permutations, and the permutation $\pi$ does not affect the state $q_k$ in the middle of the run). This new run sees the atom $a$ twice.
\end{myexample}
\exercisepart

\mikexercise{\label{ex:simple-automata}Show that deterministic register automata can recognise languages 4 and 5 from Example~\ref{ex:data-languages}.}{
Consider language 4, i.e.~the first atom appears again. The automaton stores the first atom in its unique register and then waits for a repetition to enter an accepting sink state. Here is the picture:
\mypic{5}
Consider now language 5, i.e.~every three consecutive atoms are pairwise distinct. The automaton uses two registers to store the last two atoms. There is only one control state. Here is the picture:
\mypic{6}
The unique location is the yellow one shown above and thus different occurrences of the yellow state should be seen as self-loops. The picture depicts three kinds of self-loops in this unique control state: a self-loop which goes from zero defined registers to one defined register, a self-loop which goes from one defined register to two defined registers, and a self-loop from two defined registers to two defined registers.
}

\mikexercise{\label{ex:epsilon-transitions} Show that the expressive power of nondeterministic register automata is not affected if we allow $\varepsilon$-transitions.}{The classical construction works. 
	This is most easily seen using semantic equivariance. Let $Q$ be the states of the automaton (recall that a state consists of a location and a register valuation). Consider an automaton which uses $\varepsilon$ transitions, i.e.~it has two transition relations:
	\begin{align*}
		\underbrace{\delta_\varepsilon \subseteq Q \times Q}_{\text{$\varepsilon$-transitions}} \qquad \underbrace{\delta \subseteq Q \times (\Sigma \times \atoms) \times Q}_{\text{usual transitions}},
	\end{align*}
	both of which are semantically equivariant. It is not hard to see that if $\delta_\varepsilon$ is semantically equivariant, then the same is true for its reflexive transitive closure $\delta_\varepsilon^*$. Also, semantically equivariant relations are closed under composition, and therefore $\gamma = \delta \circ \delta_\varepsilon^*$ is semantically equivariant. We replace the original transition relation by $\gamma$. We also replace the final states by those states that can reach a final state via $\delta_\varepsilon^*$, the resulting set of final states is also equivariant. The resulting automaton has no $\varepsilon$-transitions, and it recognises the same language as the original one. 
}



\mikexercise{\label{ex:myhill-wrong-for-register}For languages of data words one can also define the Myhill-Nerode relation, as used in minimisation of deterministic automata. Show a language of data words where every deterministic register automaton that recognises the language distinguishes (by its state) some two words which are Myhill-Nerode equivalent.}{An example of such a language is $\set{abc : \mbox{$a,b,c \in \atoms$ are distinct}}$. After reading $ab$, the automaton should be in the same state as after reading $ba$. This example would go away if automata had registers that store unordered pairs of atoms. But then we could consider the following language, where addition is done modulo 3:
\begin{align*}
 \set{a_0a_1a_2 a_{i} a_{i+1} a_{i+2} : a_0,a_1,a_2 \in \atoms \mbox{ are distinct}}.
\end{align*}
To have a minimal automaton for the above language, we would need registers that store triples of atoms modulo cyclic permutations. Groups other than $\mathbb Z_3$ could also be used. In Section~\ref{sec:orbit-finite-automata}, we introduce an extension of register automata that does not suffer from the problems described in this exercise. } 



\mikexercise{Show that a nondeterministic register automaton can recognise language 2 from Example~\ref{ex:data-languages}, but a deterministic one cannot.}{Language 2 says that some data value appears twice. After reading sufficiently many letters, a deterministic register automaton will necessarily forget one of the previously read letters, in the sense that it will not be in any register. This letter can be read again. How arguments of this type should be formalised can be seen in the solution to the next exercise, Exercise~\ref{ex:guessing}.}

\mikexercise{\label{ex:guessing}Call a nondeterministic register automaton \emph{guessing} if there exists a transition $t \in \delta$ such that some data value in the target state appears neither in the source state nor in the input. Give an example of a language that needs guessing to be recognised. }{ The alphabet is $\atoms$ and the language, call it $L$, is ``the atom in the last position does not appear on other positions''. This language is the reverse of language 4. (This exercise also shows that nondeterministic automata without guessing are not closed under reverses.) With guessing, the language $L$ can easily be recognised, by simply reversing all arrows in the automaton for language 4 from Exercise~\ref{ex:simple-automata}. The guessing corresponds to this reversed arrow:
\mypic{7}
 Let us prove that $L$ is not recognised by any nondeterministic automaton without guessing. Toward a contradiction, suppose that $L$ is recognised by an automaton without guessing. Let $n$ be strictly bigger than the number of registers. The word $a_1 \cdots a_{n+1}$ consisting of $n+1$ distinct atoms belongs to the language, and hence must admit an accepting run. Decompose this accepting run as $\sigma \cdot t$ where $t$ is the last transition, which reads the letter $a_{n+1}$, and $\sigma$ is the rest of the run, which reads the letters $a_1 \cdots a_{n}$. Since the automaton is not guessing, none of the states in the run $\sigma$ contains $a_{n+1}$. Furthermore, by assumption on $n$ being greater than the number of registers, some $a \in \set{a_1,\ldots,a_n}$ does not appear in the last state of $\sigma$. Let $\pi$ be a permutation of the atoms which swaps $a$ with $a_{n+1}$. Applying $\pi$ to $\sigma$ yields a new run $\pi(\sigma)$ which has the same last state as $\sigma$, since the swapped atoms are not present in that state. Therefore, $\pi(\sigma) \cdot t$ is also an accepting run, but the word it accepts contains the last letter $a_{n+1}$ twice.
}

 
A corollary of the above two exercises is that:
\begin{align*}
\text{deterministic} \subsetneq \text{nondeterministic without guessing} \subsetneq \text{nondeterministic}.
\end{align*}
 
\mikexercise{\label{ex:weakly-guessing} Call a nondeterministic register automaton \emph{weakly guessing} if every accepting run has the following property: if 
the transition reading the $i$-th letter loads a data value $a$ into some register $r$, then $a$ appears in some position $j \ge i$ such that the transitions reading letters $\set{i,\ldots,j}$ do not remove $a$ from register $r$. Show that for every nondeterministic register automaton there is a weakly guessing one which accepts the same words. }{Instead of storing an atom that does not appear in the input before it is erased from the registers, use an undefined register with a special marker stored in the control state.}

\section{Emptiness and universality for register automata}
In this section, we discuss two decision problems for register automata:
\begin{itemize}
	\item nonemptiness (does the automaton accept at least one input word); and
	\item universality (does the automaton accept all input words).
\end{itemize}
 When talking about decidability, we assume that the transition function in a register automaton is represented according to the syntactic equivariance condition.
\begin{theorem}\label{thm:register-decidable-emptiness}Emptiness is decidable for nondeterministic register automata.
\end{theorem}
\begin{proof} This proof just sketches the decidability argument, the complexity is discussed in Exercise~\ref{ex:emptiness-complexity}.
Similarly to the orbits of transitions used in the proof of Lemma~\ref{lem:two-types-of-equivariance}, we define an \emph{orbit of states} to be a set of the form
\begin{align*}
	\set{\pi(q) : \text{$\pi$ is a permutation of the atoms}} \qquad \text{for some state $q$.}
\end{align*}
For example, if the automaton has three locations (orange, red and blue) and two registers, then there are 15 orbits of states, as shown in the following picture, with orbits represented by examples of states that use atoms $1$ and $2$:
\mypic{111}
 As in Lemma~\ref{lem:two-types-of-equivariance}, an orbit of states can be defined by saying what is the location, which registers have defined values, and what is the equality type of the atoms stored in the defined registers. Such a description takes finite space to store, and there are finitely many possible descriptions (although the number of orbits is exponential in the number of registers). The key observation is that being in the same orbit of states respects reachability, i.e.~if two states are in the same orbit, then both are reachable or both are unreachable. 	The algorithm for nonemptiness computes the orbits of reachable states. Initially, we have the orbit of the unique initial state, which has all registers undefined. If we have the equality type of some state, we can easily compute the equality types of all states reachable from it in one step; thus finishing the description of the algorithm.
\end{proof}


\begin{theorem}\label{thm:register-undecidable-universality}Universality is undecidable\footnote{This proof follows the same lines as the undecidability for a stronger model, namely timed automata, see~\cite[Theorem 5.2]{alur_theory_1994}.} for nondeterministic register automata.
\end{theorem}
\begin{proof}
We reduce from the halting problem for Turing machines, i.e.~the problem of deciding if a given Turing machine has at least one accepting computation. Suppose that we have a Turing machine which is an instance of the halting problem. We encode a computation of a Turing machine as a data word according to the following picture:
\mypic{1}
Each letter encodes a single cell in a single configuration of the Turing machine. The data word represents a sequence of configurations, padded with blanks so that they all have the same length. The configurations are separated by a letter $\#$. The labels are used to store the contents of the cell plus the control state of the head if the head happens to be over that cell. Finally, each cell gets a unique identifier, which is its data value (the same cell in consecutive configurations gets the same identifier). 
\begin{claim}
There is a nondeterministic register automaton which accepts a data word if and only if it is \emph{not} an encoding of an accepting computation of the Turing machine.	
\end{claim}
It follows that the Turing machine has no accepting computation if and only if the nondeterministic register automaton accepts all inputs. Therefore, universality of nondeterministic register automata is undecidable.
\begin{proof}[Proof of the claim]
 To prove the claim, we list the mistakes that can happen in a data word that does not encode an accepting computation of a Turing machine:
\begin{enumerate}
	\item The data values identifying the cells are chosen incorrectly. This means that:
\begin{enumerate}
\item the separator $\#$ is used with more than one data value; or
\item there exist positions $i,j$ with the same data value such that the successor positions $i+1$ and $j+1$ are defined and have distinct data values.
\end{enumerate}
The first condition can be tested using one register, the second condition using two registers.
\item There is a mistake between two consecutive configurations. Assuming that the identifiers are chosen correctly, this mistake can be tested using only one register, which is used to identify corresponding cells in consecutive configurations.
\item The first configuration is not initial, or the last configuration is not accepting. For this, no registers are needed.
\end{enumerate}
\end{proof}
\end{proof}

\exercisepart


\mikexercise{\label{ex:emptiness-complexity} The complexity of the emptiness problem for nondeterministic register automata depends on how the size $|\Aa|$ of the input automaton is measured. Show that emptiness is:
\begin{itemize}
	\item {\sc pspace}-complete if $|\Aa|$ is the number of locations and registers;
	\item {\sc np}-complete if $|\Aa|$ is the number of reachable orbits of states;
	\item polynomial time if $|\Aa|$ is the number of orbits of transitions.
\end{itemize}
}{
\begin{enumerate}
		\item
		\begin{itemize}
			\item \textsc{pspace} membership. A nondeterministic \textsc{pspace} algorithm can guess the accepted word. If the automaton has $n$ registers, then data values that are numbers $\set{0,\ldots,2n}$ are enough.
			\item \textsc{pspace} hardness. The problem is already hard for automata which ignore the input letters in the sense that acceptance for a word is uniquely determined by its length. If the state space is $n$-tuples of atoms, then an arbitrary vector of $n-1$ bits can be encoded by the pattern in which the coordinates $2,\ldots,n$ are equal to the first coordinate. Therefore, one can think of the state as coding vector of $n-1$ bits, which can be used to store the tape contents of a Turing machine. A quantifier-free formula of size polynomial in $n$ can be used to describe the transitions of the machine.
		\end{itemize} 
		\item
		\begin{itemize}
			\item \textsc{np} hardness. We reduce from the following problem: given a formula
			\begin{align*}
 \varphi(a_1,\ldots,a_n,b_1,\ldots,b_n)
\end{align*} which is a Boolean combination of equalities and inequalities, decide if there is a satisfying assignment where all $a_i$ are pairwise different and all $b_i$ are pairwise different. This is an \textsc{np}-hard problem because the pattern of equalities between $\bar a$ and $\bar b$ can be used to encode an arbitrary vector of $n$ bits (say that bit $i$ is true if and only if the vectors $\bar a$ and $\bar b$ agree on coordinate $i$). The above problem is at least as hard as emptiness for register automata, even when there are three orbits of reachable states. Indeed, suppose that the automaton has two locations $\ell_0$ and $\ell_1$, one initial and final, and three orbits of reachable configurations:
\begin{align*}
\underbrace{ \ell_0(\bot,\ldots,\bot)}_{\text{orbit 1}} \qquad \underbrace{\ell_0(\overbrace{a_1,\ldots,a_n}^{\text{distinct data values}})}_{\text{orbit 2}} \qquad \underbrace{\ell_1(\overbrace{b_1,\ldots,b_n}^{\text{distinct data values}})}_{\text{orbit 3}}
\end{align*} The formula $\varphi$ could be used as a guard in a transition that goes from the second orbit to the third orbit. 			
			\item \textsc{np} membership.
			Consider a graph, where the vertices are orbits of states, and there is an edge from orbit $Q_1$ to orbit $Q_2$ if and only if there is some transition from some state in $Q_1$ to some state in $Q_2$. Because the automaton is equivariant, the following conditions are equivalent 
			\begin{enumerate}
				\item There exists a state $q_1$ in orbit $Q_1$ and a state $q_2$ in orbit $Q_2$ such that some transition leads from $q_1$ to $q_2$ in one step.
								\item For every state $q_1$ in orbit $Q_1$ there exists a state $q_2$ in orbit $Q_2$ such that some transition leads from $q_1$ to $q_2$ in one step.
			\end{enumerate}
			It follows that the automaton is nonempty if and only if the graph described above contains a path from some orbit in the initial states to some orbit in the accepting states. Necessarily such a path has length bounded by the number of orbits. By testing quantifier-free formulas for satisfiability, one can test this in \textsc{np}.			
		\end{itemize}
		\item \textsc{ptime} membership. We do the same argument as in \textsc{np} membership, only this time the edges of the graph can be computed in polynomial time.
	\end{enumerate}}


\mikexercise{\label{ex:minsky}\label{ex:one-register-undec}The undecidability proof in Theorem~\ref{thm:register-undecidable-universality} used automata with two registers but no guessing (as defined in Exercise~\ref{ex:guessing}). Show that, in the presence of guessing, universality remains undecidable even with one register.}{
We prove that the (non-)halting problem for Minsky machines reduces to this universality. Recall that a Minsky machine has a finite set of states, two counters storing natural numbers, and a set of transitions which can increment the counters, decrement them and test them for zero. It is an undecidable problem to decide, given a Minsky machine and two control states $p,q$, if the machine admits a run that goes from $p$ with both counters empty to $q$ with both counters empty. We can view a run of a Minsky machine as a sequence which alternates between control states and counter operations, in a way consistent with the transition relation, as in the following picture:
\mypic{2}
The counter operations are valid if for every counter $c \in \set{\text{A, B}}$, one can pair (the arcs in the picture above) the increments and decrements on counter $c$ such that the increment comes before, and there is no zero test in between. Such a run with a pairing can be encoded as a data word, by adding a unique data value for each arc and using some special data value for positions that are not on arcs (i.e.~states or zero tests), as in the following picture:
\mypic{3}
We claim that a nondeterministic automaton with one register (but with guessing) can recognise the set of data words that are not the encoding of an accepting run with a pairing, and hence undecidability of universality follows in the same way as in Theorem~\ref{thm:register-undecidable-universality}. The most interesting type of problem is that some arc is wrong: for this the automaton guesses some atom $a$ at the beginning, and checks that this atom is either not used exactly two times, or the first use is not an increment, or the second use is not a decrement of the same counter, or in between there is a test for zero on the appropriate counter.}



\section{Alternating register automata}
\label{sec:alternating-automata}
In a nondeterministic automaton, each transition is chosen nondeterministically in favour of acceptance, i.e.~for acceptance it suffices that there is at least one choice of transitions that gives an accepting run. An alternating automaton is a generalisation of a nondeterministic automaton, where the syntax specifies which locations choose transitions in favour of acceptance, and which locations choose transitions against acceptance. The main result of this section is that emptiness is decidable for a restricted version of alternating register automata\footnote{The main result of Section~\ref{sec:alternating-automata}, namely decidability of emptiness for alternating one-way register automata with one register, was first shown in~\cite{DBLP:journals/tocl/DemriL09}. A tree extension of the result can be found in~\cite{DBLP:journals/tocl/JurdzinskiL11}.}. 


 
\paragraph*{Alternating register automata.}
The syntax of an \emph{alternating register automaton} is defined the same way as for a nondeterministic register automaton, except that there is an additional partition of the locations into two parts, called \emph{existential} and \emph{universal}. 

We define the semantics of the automaton using \emph{bags}\footnote{An alternative but equivalent semantics would use a game played by two players, called ``universal'' and ``existential''.}, where a bag is defined to be a set of states. Here is a picture of a bag:
\mypic{65}
	We write $P,Q$ for bags. Bags can be infinite, but for the automata that we will mainly be interested in -- non-guessing ones -- only finite bags will play a role. If $a$ is an input letter (consisting of a label and a data value) and $P,Q$ are bags then we write
	\begin{align*}
		P \stackrel a \to Q	
	\end{align*}
	 if the following conditions hold: 
\begin{itemize} 
	\item for every state $p \in P$ with an existential location, the bag $Q$ contains some state $q$ such that $(p,a,q)$ is a transition; and
	\item for every state $p \in P$ with an universal location, the bag $Q$ contains all states $q$ such that $(p,a,q)$ is a transition. 
\end{itemize} 
A data word $a_1 \cdots a_n$ is accepted by an alternating automaton if there exists a run, which is defined to be a sequence of bags
\begin{align*}
\overbrace{\set{\text{initial state}}}^{\text{initial bag}}= Q_0 \stackrel {a_1} \to Q_1 \stackrel {a_2} \to \cdots \stackrel {a_{n}} \to Q_n 
\end{align*}
where the last bag is accepting, in the sense that all of its states use accepting locations. 
 We define $\to$ to be the union of all relations $\stackrel a \to$, ranging over all letters $a$. In terms of this notation, an alternating automaton is nonempty if and only if some accepting bag is reachable from the initial bag via a finite number of steps using the relation $\to$.

Languages recognised by alternating register automata are closed under complementation, essentially by design, see Exercise~\ref{ex:alternating-complement}.




\paragraph*{An emptiness algorithm using well quasi-orders.} Nondeterministic register automata are the special case of alternating register automata where all states are existential.
By Exercise~\ref{ex:alternating-complement}, the emptiness and universality problems for alternating register automata are essentially the same problem, which is undecidable by Theorem~\ref{thm:register-undecidable-universality}. 

We now identify a restriction on alternating automata which makes emptiness (or universality) decidable. 
We consider alternating automata that are \emph{non-guessing}, see Exercise~\ref{ex:guessing}, which means that for every transition $(p,a,q)$, each atom that appears in $q$ must appear in either $p$ or $a$. 
By the remarks in Exercise~\ref{ex:one-register-undec}, universality is undecidable for non-guessing nondeterministic automata with two registers, or even guessing nondeterministic automata with one register. Therefore, emptiness is undecidable for alternating automata that are either non-guessing with two registers, or guessing with one register. Anything below that is decidable:
 
 \begin{theorem}\label{thm:one-register-alternating}
 	Emptiness is decidable for alternating non-guessing automata with one register.
 \end{theorem}
 
 The rest of this section is devoted to proving the above theorem. Nonemptiness is semi-decidable, i.e.~there is an algorithm (guess a word and run the automaton on it) which terminates if and only if the input automaton is nonempty. Therefore, in order to prove decidability it suffices to show that emptiness is also semi-decidable. The rest of this section is devoted to designing an algorithm which inputs an automaton (alternating, non-guessing, and with one register) and terminates if and only if the input automaton is empty. In other words, we are searching for a finite and computable witness of emptiness. 

 Fix an alternating non-guessing automata with one register. Because the automaton is non-guessing, only finite bags can be reached from the initial state. Therefore, from now on, all bags are assumed to be finite.

 
 As in the definition of semantic equivariance from Section~\ref{sec:register-automata}, permutations of the atoms can be applied to states and to bags of states. The following order on bags is the key to our proof: we write $P \le Q$ if there is some permutation of the atoms $\pi$ such that $P \subseteq \pi(Q)$. Here is a picture:
\mypic{13}
 The relation $\le$ is easily seen to be a quasi-ordering, i.e.~it is transitive and reflexive, but not necessarily anti-symmetric. 
 Call a set of bags \emph{upward closed} if whenever it contains a bag $P$, and $P \le Q$, then it also contains $Q$. The \emph{upward closure} of a set of bags is the least upward closed set of bags that contains it. 
 To show that emptiness is semi-decidable, we will use upward closed invariants as described in the following lemma.
 \begin{lemma}\label{lem:upward-closed-invariant}
An alternating non-guessing automata with one register is empty if and only if there is an upward closed invariant, i.e.~a family of bags $\Qq$ which:
\begin{enumerate}
	\item is upward closed; and
	\item \label{it:no-accepting-bags} contains no accepting bags; and
	\item contains the initial bag; and
	\item is closed under transitions, i.e.~$P \to Q$ and $P \in \Qq$ imply $Q \in \Qq$.
\end{enumerate}
\end{lemma}
\begin{proof}
	Clearly, if there is an upward closed invariant, then the automaton is empty. For the converse implication, suppose that the automaton is empty, and define $\Qq$ to be the bags that cannot reach an accepting bag. By definition, $\Qq$ contains no accepting bags, and by assumption on emptiness, $\Qq$ contains the initial bag. Also, $\Qq$ is closed under transitions. To prove the lemma, it remains to show that $\Qq$ is upward closed. This will follow from the following property of the transition relation. 
	
	\begin{claim}\label{lem:wts} The order $\le$ on bags is compatible with the transition relation~$\to$ in following sense: for every transition $P \to Q$ and every $P' \le P$ there exists some $Q' \le Q$ with $P' \to Q'$.
	\end{claim}
	\begin{proof}
	Here is the picture of compatibility: \mypic{17}
	Because $\to$ is closed under atom permutations, and also closed under making the first argument a smaller bag.
	\end{proof}

	Since the family of accepting bags is downward closed, a corollary of compatibility as in the above claim is that if a bag can reach an accepting bag, then the same is true for any smaller bag. The contrapositive is that if a bag belongs to $\Qq$, then the same is true for any bigger bag. 
\end{proof}

The general idea behind the semi-algorithm for emptiness is to search for upward closed invariants as in the above lemma. To represent these invariants in a finite way, we will use the following result. 
\begin{lemma}\label{lem:invariant-finitely-generated}
	Every upward closed invariant is the upward closure of some finite family of bags.
\end{lemma}

Before proving the above lemma, we use it to complete the proof of Theorem~\ref{thm:one-register-alternating}. This semi-algorithm for emptiness works as follows: searches through all finite families of bags, and terminate with success if there is a finite family whose upward closure is an upward closed invariant in the sense of Lemma~\ref{lem:upward-closed-invariant}. By Lemmas~\ref{lem:upward-closed-invariant} and~\ref{lem:invariant-finitely-generated}, this semi-algorithm terminates with success if and only if the automaton is empty. It remains to show how one can check, given a finite family of bags, if its upward closure is an upward closed invariant. The first condition, about upward closure, is vacuously satisfied. The second condition, about having no accepting bags, corresponds to checking that the finite family has no accepting bags because upward closure cannot add accepting bags. The third condition, about containing the initial bag, corresponds to checking if the finite family contains either the initial bag or the empty bag (which is the unique bag that is strictly smaller than the initial bag). Finally, we are left with checking if the upward closure is closed under transitions. By compatibility, see Claim~\ref{lem:wts}, the upward closure of a finite family $\Qq$ is closed under taking transitions if and only if for every bag $Q \in \Qq$ and every transition $Q \to P$, the target bag is in the upward closure of $\Qq$, which can easily be checked by enumerating all finitely many candidates for the transition $Q \to P$. Closure under transitions is illustrated in the following picture: \mypic{15} 
The idea is that the orange area, i.e.~the upward closure of $\Qq_0$, is a trap in the sense that no transition can leave the orange area.

It remains to prove Lemma~\ref{lem:invariant-finitely-generated}, which we do in the rest of this section. This is done using well quasi-orders. We say that a quasi-order is a \emph{well quasi-order} if it is well-founded (no infinite strictly decreasing chains) and has no infinite antichains. The technique of well quasi-orders\footnote{The well quasi-order technique was independently introduced in~\cite{DBLP:journals/iandc/AbdullaCJT00} and~\cite{DBLP:journals/tcs/FinkelS01}, and is currently known as the technique of \emph{well-structured transition systems}.}, as used in the following proof, is a common method of proving decidable properties for systems with infinitely many configurations.
 
\begin{lemma}\label{lem:wqo-finite-basis}
	In a well quasi-order, every upward closed set is the upward closure of a finite set.
\end{lemma}
\begin{proof}
By well-foundedness, every upward closed set is the upward closure of its minimal elements. The minimal elements form an antichain, and hence there can only be finitely many of them (up to the equivalence where $x$ and $y$ are equivalent if $x \le y$ and $y \le x$).
\end{proof}

To prove Lemma~\ref{lem:invariant-finitely-generated}, and therefore also Theorem~\ref{thm:one-register-alternating}, it is enough to establish that the relation $\le$ on bags is a well quasi-order. 
\begin{lemma}\label{lem:bag-is-wqo}
	The relation $\le$ on bags is a well quasi-order.
\end{lemma}
\begin{proof} 
It is clear that the relation is well-founded, since a strict decrease on bags implies a strict decrease in the cardinality (recall that we only consider finite bags). It remains to show that there is no infinite antichain. Define 
	\begin{align*}
		 	\xymatrix@C=2cm{
			\text{bags} \ar[r]^<<<<<<<<<<<{\text{\small profile}} & 	\set{\text{true, false}}^{\text{ locations}} \times \Nat^{\text{ nonempty subsets of locations}}
		}
	\end{align*}
	to be the function
	which maps a bag to the information explained in the following picture:
	\mypic{112}
	The profile mapping reflects the order in the following sense 
	\begin{align}\label{eq:profile-monotone}
		\text{profile}(P) \le \text{profile}(Q) \quad \text{implies} \quad P \le Q,
	\end{align}
	where the order on profiles is coordinate-wise, with false $\le$ true (see Exercise~\ref{ex:right-order-on-profiles} for why the converse implication fails). Because the order is reflected, the profile mapping maps antichains of bags to antichains of profiles. Since there are no infinite anitichains of profiles by Exercise~\ref{ex:wqo-dixon}, it follows that there are no infinite antichains of bags.
\end{proof}


 



\paragraph*{The general technique.}
Using the same proof, we obtain the following generalisation of Theorem~\ref{thm:one-register-alternating}.
\begin{theorem}\label{thm:wts}
The following problem is decidable.
\decisionproblem{
	\begin{itemize}
		\item A directed graph where every node has finite outdegree;
		\item A well quasi-order $\le$ on vertices which satisfies the following condition:
	\mypic{114}
		\item A source vertex plus a set of target vertices that is downward closed.
	\end{itemize}
	The input is represented by algorithms for: enumerating the vertices, testing membership in the target set, testing the well quasi-order, and computing the neighbour list of a given vertex.
}
{
	Is there a path from the source to one of the targets?
}
\end{theorem}



\paragraph*{A temporal logic for data words.} 
One register alternating automata can be dressed up in the syntax of a temporal logic. The idea is to add one register to linear temporal logic {\sc ltl}. We do not give the detailed syntax and semantics, only some examples. We are extending {\sc ltl}, so we can write a formula 
\begin{align*}
 a \ \mathsf{until}\ b,
\end{align*}
which is true in a (data) word if there is some position with label $b$ such that all earlier positions have label $a$. Instead of $a,b$ we could have used previously defined formulas, and Boolean combinations are also allowed. There is also an operator $\mathsf{next}$ to access the next position. For example, the formula 
\begin{align*}
 \underbrace{(a \lor \neg a)}_{\top} \ \mathsf{until}\ (a \land \ \mathsf{next} \ a)
\end{align*}
says that there exist two consecutive positions with label $a$. We use $\mathsf{finally}\ \varphi$ as syntactic sugar for $\top\ \mathsf{until}\ \varphi$. If we only use the operators $\mathsf{until}$ and $\mathsf{next}$, then we have exactly the logic {\sc ltl}, which is insensitive to the data values. To access the data values, we can add an operator $\mathsf{store}$ which stores the current data value and a formula $\mathsf{same}$ which is true whenever the current value is equal to the stored one. For example, the formula
\begin{align*}
 \mathsf{store}\big( \mathsf{next}\ \neg \big(\mathsf{finally \ same} )\big)
\end{align*}
says that the first data value does not repeat, i.e.~after storing it one cannot find the same one again. In principle, we could have several different registers for storing data values, but if we want to translate the logic to one register alternating automata, then only one register is allowed (and hence there is no need to give it a name). The register can be reused, e.g.~the following formula says that whenever the first data value of the word is used, then the next two positions have distinct data values:
\begin{align*}
 \mathsf{store}\big( \mathsf{next}\ \neg \big(\mathsf{finally} (\mathsf{same} \land \mathsf{next}\ (\mathsf{store} (\mathsf{next} \ \mathsf{same}) ) )\big) \big)
\end{align*}
Every formula of this temporal logic can be converted into an alternating one register automaton, and therefore one can decide if a formula is true in at least one data word.

%\mikexercise{Consider a one register alternating automaton over infinite words, with a safety acceptance condition, which means that a runs are infinite sequences of bags (and there is no requirement that some configuration is reached in infinitely often or at least once). Show an automaton which recognises the language
%\begin{align*}
% \set{w_1 \# w_2 \# \cdots : w_1,w_2,\ldots \in \atoms^* \mbox{ and all but finitely many of the words $w_i$ are equal}}
%\end{align*}
%
%}{}
%It will be easier to describe the complement. 
%\begin{itemize}
%	\item For infinitely many $i$, some data value appears in $w_i$ but not in $w_{i+1}$;
%	\item One can execute the following program
%	\begin{enumerate}
%		\item Go to some position $x$;
%		\item Find some position $y > x$ such that there is no $\#$ between $x$ and $y$
%		\item With $y$ as in the previous step, find some position $x > y$ such that $y,x$ have the same data value;
%		\item Goto step 2.
%	\end{enumerate}
%\end{itemize}

\exercisepart


\mikexercise{\label{ex:alternating-complement}Show that languages recognised by alternating register automata are closed under complementation.}{Let $\Aa$ be an alternating register automaton, and define the \emph{dual} of $\Aa$ to be the same automaton but where we swap universal locations with existential locations, and we swap accepting locations with nonaccepting locations. We claim that $\Aa$ accepts a word if and only if its dual rejects. 

We prove that for every state $q$ and input data word $w$, the automaton $\Aa$ accepts $w$ starting in the bag $\set q$ if and only if the dual rejects $w$ starting in $\set q$. The proof is by induction on the length of the input. For the induction base of empty inputs, we use the fact that accepting and nonaccepting locations are swapped. Let us do the induction step. Suppose that the input is $aw$ for some letter $a$ and remaining input $w$. If the state $q$ uses an existential location, then saying that $\Aa$ accepts $aw$ from $q$ means that there is some transition $(q,a,p)$ such that $\Aa$ accepts $w$ from $p$. By the induction assumption, the dual rejects $w$ from $p$. Since $q$ is universal in the dual, it follows that the dual rejects $aw$ from $q$, since there is some transition which leads to rejection. The case when $q$ uses a universal state in $\Aa$ is done the same way.
}



\mikexercise{\label{ex:wqo-monot}Show that a quasi-order is a well-quasi-order if and only if every infinite sequence contains a monotone subsequence, i.e.~one where $i \le j$ implies $x_i \le x_j$.}
{The right-to-left implication is immediate because infinite antichains and infinite strictly decreasing sequences are both examples of infinite sequences without infinite monotone subsequences. Let us prove the remaining implication, i.e.~in a well quasi-order every infinite sequence has an infinite monotone subsequence.

Let $x_1,x_2,\ldots$ be some sequence in a well quasi-order. Consider the set of minimal elements that appear in the sequence. This set must be finite up to equivalence in the quasi-order, otherwise we would have an infinite antichain. Furthermore, for every element in the sequence there must be some smaller or equal element that is minimal, since otherwise we would have an infinite strictly decreasing sequence. Cut off a finite prefix of the sequence where all minimal elements are found up to equivalence, and reapply the argument, and continue doing this forever. In the limit we get a partition of the sequence into finite factors
\begin{align*}
 \underbrace{x_1, \ldots, x_{i_1}}_{\text{factor 1}}, \underbrace{x_{i_1+1}, \ldots, x_{i_2}}_{\text{factor 2}}, \ldots
\end{align*}
such that every element from outside the first factor is greater or equal to some element that appears in the previous factor. We can view this factorisation as a directed acyclic graph on the indices $\set{1,2,\ldots}$ which has an edge from $i$ to $j$ if $x_i \le x_j$ and $i,j$ are in consecutive factors. This directed acyclic graph has finite degree because factors are finite, and it has arbitrarily long paths. Therefore, it must have an infinite path by K\"onig's lemma.

 Another solution uses Ramsey's Theorem. Take some infinite sequence $x_1,x_2,\ldots$ and colour each pair $i < j$ with ``smaller'', ''bigger or equal'' or ``incomparable'', depending on the relationship of $x_i$ and $x_j$. By Ramsey's theorem, there is an infinite subsequence where all pairs get the same colour. This colour has to be ``bigger or equal'', since the other possibilities would imply an infinite antichain or descending sequence.}

%
% has an infinite monotone subsequence, i.e.~a subsequence $x_1,x_2,\ldots$ where $x_i \le x_j$ holds whenever $i \le j$. }{An infinite antichain or a violation of well-foundedness are both examples of sequences without infinite monotone subsequences, which shows the right-to-left implication. It remains to show the converse implication, i.e.~that in a well-quasi order every infinite sequence has an infinite monotone subsequence.
%
%We use the following observation: in a well-quasi order every downward closed set has 
%
%Define a \emph{bad sequence} to be an infinite sequence without monotone subsequences. We need to show that there are no bad sequences. 
%
%
%Define $B \subseteq X^*$ to be the prefixes of bad sequences and order this set lexicographically.
%Define a tree labelled by 
%Choose $x \in X$ to be a minimal element such that there is no 
%
%}
\mikexercise{\label{ex:wqo-dixon}Show that for every dimension $d \in \set{1,2,\ldots}$, the set $\Nat^d$ is a well quasi-order with respect to the coordinatewise ordering.}{Using Exercise~\ref{ex:wqo-monot} it suffices to show that every infinite sequence in $\Nat^d$ has an infinite monotone subsequence. This is shown by induction on $d$. The induction base of $d=1$ is easy to see. For the induction step, consider a sequence
\begin{align*}
 x_1,x_2,\ldots \in \Nat^{d+1}. 
\end{align*}
By the induction assumption, there is an infinite subsequence such that the projection onto the first coordinate is monotone. By induction assumption again, that subsequence has an infinite subsequence where the projection onto the remaining coordinates is monotone, and the result follows.
 }


\mikexercise{\label{ex:right-order-on-profiles} Show that the converse implication in~\eqref{eq:profile-monotone} is not true. Find a well quasi-order on profiles which turns the implication into an equivalence. }
{The counterexample is the bags
	\mypic{113} 
	which have profiles that are coordinate-wise incomparable (an equivalence in~\eqref{eq:profile-monotone} would be recovered if we considered a slightly different order on profiles, but we do not do this only the implication~\eqref{eq:profile-monotone} is needed for our reasoning). }

\mikexercise{For a possibly infinite alphabet $\Sigma$, define the Higman ordering on $\Sigma^*$ to be the relation of not necessarily connected substrings. Show that this is a well quasi-ordering. }{See~\cite[Exercise 1.10]{schmitz:cel-00727025}}

\mikexercise{Suppose that the atoms are equipped with a total order. Show that emptiness remains decidable for one register alternating automata without guessing, even when the machine can use the order to compare the register with the current data value\footnote{The paper~\cite{lasota_wqo_2018} investigates what structure on the atoms can be used so that the order $\le$ on bags is a well quasi-order.}. }{The same proof as without order. The only difference is that order-preserving bijections are used in the definition of the quasi-order, and the Higman order is used to show that this is a well quasi-order. More specifically, when proving the variant of Lemma~\ref{lem:bag-is-wqo}, instead of using vectors of natural numbers indexed by subsets of locations, we use sequences of subsets of locations, ordered by the Higman ordering.
 } 
\mikexercise{For a Turing machine with one tape, define a \emph{gain} to be the process of taking a configuration and inserting nondeterministically one new cell in some position not below the head, with any label from the work alphabet. 
Define a \emph{gainy computation step} of a Turing machine to be a finite (possibly zero) number of gains followed by a normal step of computation. Show that the halting problem is decidable for Turing machines with semantics defined using gainy computation steps. }{Using Theorem~\ref{thm:wts} and the Higman ordering on configurations. }

\mikexercise{\label{exercise:higman}Show that there is an infinite antichain for the following order on $\atoms^*$:
\begin{align*}
 w \le v \quad \mbox{if} \quad \mbox{$w$ is Higman smaller or equal to $\pi(v)$ for some permutation of $\atoms$. }
\end{align*}

 }{ This solution is by Klin and Lasota.
Let $W$ be the set of words like the one depicted on the following picture, with circles denoting consecutive data values, and dotted lines denoting equality:
\mypic{72}
Note that the atom in the first letter is special because it appears four times and all other atoms appear two times. We claim that if $w$ and $v$ are words in $W$ of different lengths, then $w$ is not in the same orbit (i.e.~equivalent up to atom permutations) as any subsequence of $v$. In other words, there is no mapping $f$ from positions of $w$ to positions of $v$ which preserves the order and equality on data values. Indeed, such a mapping would have to map the first position to the first position (because the first letter contains the special atom that appears four times), and therefore also the third position to the third position. It follows that the second position must be mapped to the second position, and therefore also the fifth position to the fifth position. 
Arguing inductively, we see that the $i$-th position needs to be mapped to the $i$-th position. In other words, $w$ needs to be mapped to a prefix of $v$. This cannot be, because, the last position of $w$ is mapped to the last position of $v$.
}

\mikexercise{Show that there is a language $L \subseteq \atoms^*$ that is upward closed under the Higman order, but is not recognised by a nondeterministic register automaton. }{Consider the set $W$ of words in the solution to Exercise~\ref{exercise:higman}. Let $W_P \subseteq W$ be the subset of words that have a prime number of different atoms. Finally, let $L$ be the upward closure of $W_P$ under the Higman order. We claim that this language is not recognised by a nondeterministic register automaton. Otherwise, such an automaton would need to tell the difference between words from $W$ that have prime and non-prime length. By choosing some non-computable set of numbers instead of the prime numbers, we can get a language that is not computable.} 
\section{Most models of register automata are inequivalent}
The goal of this section is to collect exercises which paint a depressing picture: with one exception, the only inclusions between models of register automata are the ones that trivially follow from the definitions. To have a richer landscape, we also consider the 
two-way variant of register automata, where the head of the automaton can move both ways, with the input being extended by markers on both sides\footnote{An in-depth study of various kinds of register automata can be found in~\cite{DBLP:journals/tocl/NevenSV04}, including undecidability of universality of nondeterministic one-way automata (Theorem~\ref{thm:register-undecidable-universality}).
The non-equivalence results summarised in Figure~\ref{fig:sep-langs} are originally found in~\cite{DBLP:journals/tcs/KaminskiF94,DBLP:journals/tocl/NevenSV04} and an unpublished Master's thesis in Polish~\cite{wysocki}. For a survey on automata and logic for infinite alphabets, see also~\cite{DBLP:conf/csl/Segoufin06}.}. For the purpose of this section, we assume that all models allow $\varepsilon$-transitions.


\mikexercise{\label{ex:alt-guessing}Find a language of data words that is recognised by an alternating register automaton with guessing, but not by any alternating register automaton without guessing.}{This solution comes from the Master's thesis of Tomasz Wysocki~\cite{wysocki}. Consider the following language over $\atoms$:
\begin{align*}
 \set{ a^n a_1 \cdots a_n : \text{$n \in \Nat$ and $a,a_1,\ldots,a_n \in \atoms$ are all distinct}}
\end{align*}

Let us first argue that an alternating register automaton without guessing cannot recognise the language. After reading a prefix of the form $a^n$, the bag can only have $a$ in its registers. Since there are finitely many possibilities for such bags, there must be some $n < m$ such that the set of reachable bags after reading $a^n$ is the same as the set of reachable bags after reading $a^m$. Therefore, if the automaton accepts $a^n a_1 \cdots a_n$, then it also accepts $a^m a_1 \cdots a_n$.

Let us recognise this language with guessing. An alternating automaton can easily check that a word is of the form $a^n a_1 \cdots a_m$ for distinct data values $a,a_1,\ldots,a_m$. The challenge is to check that $n=m$. Since languages recognised by alternating automata are closed under intersection, we assume that the input is of the form $a^n a_1 \cdots a_m$.

We only present the main idea using pictures.
 The automaton has three registers. A main thread of the automaton will read the first $n$ letters, and after reading the $i$-th letter it will be in a configuration with the initial state and register values $a, a_{i-1},a_i$ as in the following picture (the orange boxes represent these configurations, with the first two boxes being corner cases):
 \mypic{8}
The contents of the registers are above are guessed, but they are verified using alternation: the initial state is universal, and in each step it spawns off a parallel thread that checks if the current configuration corresponds to two consecutive data values in the future, as in this picture:
\mypic{9}
}

\mikexercise{Show that every two-way nondeterministic register automaton can be simulated by an alternating register automaton (with guessing and $\varepsilon$-transitions).}{This solution comes from the Master's thesis of Tomasz Wysocki~\cite{wysocki}. Consider a two-way nondeterministic automaton $\Aa$, where the locations are $\locations$ and the registers are $R$. For two states of this automaton 
\begin{align*}
 p,q \in \locations \times \text{register valuations}
\end{align*}
we say that a word admits a $(p,q)$-loop if there is a run of the automaton which begins in state $p$, ends in state $q$, and never tries to move to the left beyond the first position of the word. Here is the picture, note how the run is allowed to revisit the first position or the end delimiter $\dashv$ but it is not allowed to see the start delimiter $\vdash$.
\mypic{10}
The crucial point is to recognise loops: we will sketch that there is an alternating register automaton, such that if is initialised in a state that stores both $p$ and $q$, then it accepts if and only if there is a $(p,q)$-loop. Once loops are recognised, it is not difficult to simulate the two-way automaton (one needs to deal with the initial state and visiting the start marker $\vdash$.) To recognise loops, we observe that a data word admits a $(p,q)$-loop if and only if one of the following conditions holds:
\begin{itemize}
	\item There is some intermediate state $r$ such that the word admits a $(p,r)$-loop and an $(r,p)$-loop, as in this picture.
\mypic{12}
 To check this, the simulating alternating register automaton does an $\epsilon$-transition where it guesses $r$ and temporarily stores it in the registers. Next it universally branches by into threads for $(p,r)$ and $(r,q)$. Guessing is crucial because $r$ might contain data values from the future of the word, and $\epsilon$-transitions are used because the two-way automaton might revisit the first position an unbounded number of times.
	\item The loop does not revisit the first position, as in the following picture:
		 \mypic{11}
 The simulating alternating register automaton guesses the two configurations $p',q'$, subject to the transition requirement, and advances to the next position.
\end{itemize}
}



We now present a series of exercises with a more systematic study of the following models of automata: one-way deterministic and nondeterministic, two-way deterministic and nondeterministic, as well as one-way alternating with or without guessing. We assume $\varepsilon$-transitions are allowed in all models. The picture with these six models is in Figure~\ref{fig:sep-langs}. The picture shows the obvious containments which follow from the syntax as well as the less obvious containment from Exercise~\ref{ex:alt-guessing}. In the solutions to the following exercises, one is allowed to give answers conditional on open problems in complexity theory, such as~{\sc p = np}.
\begin{figure}[hbt]
\mypic{19} 
 \caption{\label{fig:sep-langs} Six classes of register automata and their combinations. Point 1 is the language: ``last letter appears only once'', while point 2 is the language ``all letters are distinct''. The remaining points 3, 4, 5 are Exercises~\ref{ex:sep1}-\ref{ex:sep3}, while Exercise~\ref{ex:sep-all} sums up the results by saying that all combinations are possible.}
\end{figure}



\mikexercise{\label{ex:sep3}Show a language that witnesses point 5 in Figure~\ref{fig:sep-langs}, possibly using conjectures about complexity classes being distinct. }{We want a language that is one-way alternating without guessing but which is not two-way nondeterministic. We use the same type of graph problem as in Exercise~\ref{ex:sep2}, except that instead of graph reachability we use alternating graph reachability. Since alternating graph reachability is complete for polynomial time, the language cannot be done by a nondeterministic two-way automaton, since otherwise nondeterministic {\sc LogSpace} would be equal to polynomial time. }

\mikexercise{\label{ex:sep-all} Show that all coloured areas in Figure~\ref{fig:sep-langs} contain languages.}{Suppose that 
$L$ is a language that can be done by model A but not B, and $K$ is a language that can be done by model A but not C. Define $L \& K$ to be the concatenation of $L$ and $K$ separated by a fresh symbol. As long as A, B, C are one of the six models in the figure, then the language $L \& K$ can be done by model A but neither by B or C. Using this idea, we can find examples for all coloured areas as in the following picture:
\mypic{20}}


