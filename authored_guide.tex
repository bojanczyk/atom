% authored_guide.tex
% 2011/06/23, v3.1 gamma
%
% Adapted by Diana Gillooly and David Tranah
% from Ali Woollatt's original documentation for cambridge7A

\NeedsTeXFormat{LaTeX2e}[1996/06/01]

 \documentclass{cambridge7A}
% \documentclass[spanningrule]{../cambridge7A}% option

 \usepackage{natbib}
% \usepackage[numbers]{natbib}% option

 \usepackage[figuresright]{rotating}
 \usepackage{floatpag}
 \rotfloatpagestyle{empty}

% \usepackage{amsmath}% if you are using this package,
 % it must be loaded before amsthm.sty
 \usepackage{amsthm}
 \usepackage{graphicx}

 \usepackage{txfonts}
% \usepackage[scaled=0.9]{couriers}% use if you're using \tt fonts

% indexes
% uncomment the relevant set of commands

% for a single index
% \usepackage{makeidx}
% \makeindex

% for multiple indexes using multind.sty
 \usepackage{multind}\ProvidesPackage{multind}
 \makeindex{authors}
 \makeindex{subject}

% for glossary entries
 %\usepackage[style=list]{glossary}
 \usepackage[number=none]{glossary}
\makeglossary

% theorem definitions
% see chapter 3 for details
 \theoremstyle{plain}% default
 \newtheorem{theorem}{Theorem}[chapter]
 \newtheorem{lemma}[theorem]{Lemma}
 \newtheorem{proposition}[theorem]{Proposition}
 \newtheorem{corollary}[theorem]{Corollary}
 \newtheorem{conjecture}[theorem]{Conjecture}

 \newtheorem*{theorem*}{Theorem}
 \newtheorem*{lemma*}{Lemma}
 \newtheorem*{proposition*}{Proposition}
 \newtheorem*{corollary*}{Corollary}
 \newtheorem*{conjecture*}{Conjecture}

 \theoremstyle{definition}
 \newtheorem{definition}[theorem]{Definition}
 \newtheorem{myexample}[theorem]{myexample}
 \newtheorem{prob}[theorem]{Problem}
 \newtheorem{remark}[theorem]{Remark}
 \newtheorem{notation}[theorem]{Notation}
 \newtheorem{exer}[theorem]{Exercise}

 \newtheorem*{definition*}{Definition}
 \newtheorem*{example*}{myexample}
 \newtheorem*{prob*}{Problem}
 \newtheorem*{remark*}{Remark}
 \newtheorem*{notation*}{Notation}
 \newtheorem*{exer*}{Exercise}

% \hyphenation{docu-ment baseline-skip polar}

% for this documentation, table of contents lists up to subsection level
 \setcounter{tocdepth}{2}

 \newcommand\cambridge{cambridge7A}

% remove the dot and change default for enumerated lists
 \def\makeRRlabeldot#1{\hss\llap{#1}}
 \renewcommand\theenumi{{\rm (\roman{enumi})}}
 \renewcommand\theenumii{{\rm (\alph{enumii})}}
 \renewcommand\theenumiii{{\rm (\arabic{enumiii})}}
 \renewcommand\theenumiv{{\rm (\Alph{enumiv})}}

%%%%%%%%%%%%%%%%%%%%%%%%%%%%%%%%%%%%%

% \includeonly{06authored}

%%%%%%%%%%%%%%%%%%%%%%%%%%%%%%%%%%%%%

\begin{document}

 \title[Subtitle, If You Have One]
 {Preparing Authored Books Using the \cambridge\ Class File}
 \author{Cambridge University Press\\[3\baselineskip]
 This guide was compiled using \hbox{\cambridge.cls \version}\\[\baselineskip]
 The latest version can be downloaded from:
 https://authornet.cambridge.org/information/productionguide/
 LaTeX\_files/\cambridge.zip}

 \bookabstract{This is the guide for authors who are preparing written,
 rather than edited, books.}
 \bookkeywords{\LaTeX; authored books; CUP style; cambridge7A.cls.}

 \frontmatter
 \maketitle
 \tableofcontents
% \listofcontributors

 \chapter*{Preface}

This book is about algorithms that run on objects that are infinite, but finite up to certain symmetries.
Under a suitably chosen notion of symmetry, such objects -- called \emph{orbit-finite sets} -- can be represented, searched and processed just like finite sets. The goal of the book is to explain orbit-finiteness and demonstrate its usefulness. Most of the examples of orbit-finite sets are taken from automata theory, since this is where orbit-finite sets began. 


 \mainmatter
 \label{partpage}\part{The First Part}
\include{01authored}% Introduction and basic design elements
 \include{02authored}% Numbering and headings
 \include{03authored}% Mathematics
 \include{04authored}% Figures and tables
 \include{05authored}% Reference and bibliography lists
 \include{06authored}% Indexes
 \include{07authored}% Exercises

 \backmatter
 \appendix
% if you only have one appendix, use \oneappendix instead of \appendix
 % theorem.tex
% 2011/02/28, v3.00 gamma

\chapter{amsthm commands}
\label{theorem}

You can copy and paste the following code into your root file.
Assuming you have included \verb"amsthm.sty", it will number your theorems,
definitions, etc. in a single sequence within your chapter,
e.g.~Definition~4.1, Lemma~4.2, Lemma~4.3, Proposition~4.4, Corollary~4.5.

\begin{smallverbatim} %don't copy this line!
 \theoremstyle{plain}% default
 \newtheorem{theorem}{Theorem}[chapter]
 \newtheorem{lemma}[theorem]{Lemma}
 \newtheorem{proposition}[theorem]{Proposition}
 \newtheorem{corollary}[theorem]{Corollary}
 \newtheorem{conjecture}[theorem]{Conjecture}

 \newtheorem*{theorem*}{Theorem}
 \newtheorem*{lemma*}{Lemma}
 \newtheorem*{proposition*}{Proposition}
 \newtheorem*{corollary*}{Corollary}
 \newtheorem*{conjecture*}{Conjecture}

 \theoremstyle{definition}
 \newtheorem{definition}[theorem]{Definition}
 \newtheorem{myexample}[theorem]{myexample}
 \newtheorem{prob}[theorem]{Problem}
 \newtheorem{remark}[theorem]{Remark}
 \newtheorem{notation}[theorem]{Notation}
 \newtheorem{exer}[theorem]{Exercise}
 \newtheorem{criterion}[theorem]{Criterion}
 \newtheorem{algorithm}[theorem]{Algorithm}
 \newtheorem{claim}[theorem]{Claim}

 \newtheorem*{definition*}{Definition}
 \newtheorem*{example*}{myexample}
 \newtheorem*{prob*}{Problem}
 \newtheorem*{remark*}{Remark}
 \newtheorem*{notation*}{Notation}
 \newtheorem*{exer*}{Exercise}
 \newtheorem*{criterion*}{Criterion}
 \newtheorem*{algorithm*}{Algorithm}
 \newtheorem*{claim*}{Claim}

 \newtheorem*{note}{Note}
 \newtheorem*{summary}{Summary}
 \newtheorem*{acknowledgement}{Acknowledgement}
 \newtheorem*{conclusion}{Conclusion}
\end{smallverbatim} %don't copy this line!

\endinput
 \include{root}
 % appnum.tex
% 2011/02/28, v3.00 gamma

\chapter{Numbering in appendices}
\label{appnum}

In appendices equations are numbered in one sequence by chapter,
but now the chapter `number' is an upper-case letter.
Here are examples.
\begin{equation}\label{einsteinapp}
E = Mc^2 .
\end{equation}
Here are examples of a multiline displays:
\begin{equation}
\left.\begin{array}{rcl}
x &=& a+b\\
y &=& c+d\\
z &=& e+f.
\end{array}\right\}
\end{equation}
\begin{eqnarray}
x&=&a+a+b+b\nonumber\\
&=&2a+b+b\nonumber\\
&=&2a+2b.
\end{eqnarray}

Figures and tables are similarly numbered, via the \verb"\caption", command in the usual way:
\begin{figure}[ht]
\caption{A figure in an appendix.}
\end{figure}
 \endappendix
 \addtocontents{toc}{\vspace{\baselineskip}}

% the following lines will give you references at the end of the book
 \renewcommand{\refname}{Bibliography}% if you prefer this heading
 \bookreferences % if you already have references at the end of chapters,
 % you will need this command to start a new \chapter* heading
 \bibliography{percolation}\label{refs}
 \bibliographystyle{cambridgeauthordate}

 \cleardoublepage

% end notes, if you have them
% \theendnotes


% glossary
 \printglossary

% indexes

% for a single index
% \printindex

% for multiple indexes using multind.sty
 \printindex{authors}{Author index}
 \printindex{subject}{Subject index}

\end{document}
