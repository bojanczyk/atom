\begin{theorem}\label{thm:computable-fraisse}
	Let $\structclass$ be a \fraisse class over a finite vocabulary with decidable membership. Then the \fraisse limit:
	\begin{enumerate}
\item is oligomorphic; and
\item has a computable Ryll-Nardzewski function; and
\item has a decidable first-order theory.
		\end{enumerate}
\end{theorem}
\begin{proof}
	Oligomorphism of the \fraisse limit follows from Theorem~\ref{thm:fs-qf}. 
	
	 Consider the Ryll-Nardzewski function. By homogeneity, two tuples in the \fraisse limit are in the same orbit if and only if their generated substructures are the same. It follows that the number of orbits of $k$-tuples is the same as the number of isomorphism types of structures in $\structa$ with $k$ generators. The latter number can be computed by the assumption on $\structclass$. 
	
	We are left with deciding the first-order theory with parameters. 
	
	We begin by explaining how elements of the \fraisse limit, call it $\structh$, can be represented in a finite way. Recall that the \fraisse limit is constructed, in Lemma~\ref{lem:fraisse-surjective}, as the limit of a sequence of structures
	\begin{align*}
		\structh_1 \subseteq \structh_2 \subseteq \cdots.
	\end{align*}
We now argue that this construction is effective. 
	\begin{claim}
		Given $n$, one can compute $\structh_n$.
	\end{claim}
	\begin{proof}
		An inspection of the proof in Lemma~\ref{lem:fraisse-surjective} shows that, in order to compute the structure $\structh_n$, one needs the following assumptions on $\structclass$: (a) the class $\structclass$ is recursively enumerable; and (b) given an instance of amalgamation, one can compute a solution. To prove the claim, we show that both (a) and (b) follow from our assumption on $\structclass$. For (a), we first enumerate all structures with 1 generator, then all structures with 2 generators, and so on. For (b), we observe that if an instance of amalgamation uses structures of size at most $k$, then the solution is generated by at most $2k$ elements, and therefore the solution can be found by exhaustive search.
	\end{proof}
 Thanks to the above claim, we can use the following finite representation for elements in the \fraisse limit: each element is represented as a pair $(n,a)$ where $n \in \set{1,2,\ldots}$ is such that $a$ appears in $\structh_n$ for the first time. For elements represented this way, we can evaluate quantifier-free formulas, since a quantifier-free formula can be evaluated in $\structh_n$ with $n$ chosen large enough so that it covers all arguments. 

 It remains to show that the first-order theory is decidable. To do this, we show that for every first-order formula not only an equivalent quantifier-free formula exists (which follows from Theorem~\ref{thm:fs-qf}), but also this formula can be computed. It is enough to eliminate one quantifier
\begin{align*}
	\exists x \underbrace{\varphi(x_1,\ldots,x_n,x)}_{\text{quantifier free}}.
\end{align*}
Consider the structures in $\structclass$, along with distinguished elements corresponding to the free variables, which satisfy the formula $\varphi$:
\begin{align*}
	\structclass_\varphi = \set{(\structa,\overbrace{a_1,\ldots,a_n}^{\bar a},a) : \structa \in \structclass \text{is generated by $\bar a a$ and }\structa \models \varphi(\bar aa)}.
\end{align*}
Up to isomorphism, the above set is finite and can be computed thanks to the assumption on $\structclass$. 
Because the \fraisse limit is homogeneous, a tuple $\bar aa$ in the \fraisse limit satisfies $\varphi$ if and only if $\structclass_\varphi$ contains the substructure generated by $\bar a$ (together with the distinguished $\bar a$). Define 
$\structclass_{\exists x\varphi}$ to be the following projection of $\structclass_\varphi$: for each $(\structa,\bar aa) \in \structclass_\varphi$, remove the last component $a$ from the valuation and keep only those elements of $\structa$ that are generated by $\bar a$. A tuple $\bar a$ in the \fraisse limit satisfies the quantified formula $\exists x \varphi$ if and only if $\structclass_{\exists x\varphi}$ contains the substructure generated by $\bar a$ (together with the distinguished $\bar a$). This property can be expressed using a quantifier-free formula.
\end{proof}

All \fraisse classes discussed in this chapter satisfy the assumptions of the above theorem, in particular:
\begin{enumerate}
	\item finite sets with equality only (Example~\ref{example:pure-set-fraisse});
	\item finite directed and undirected graphs (Example~\ref{example:directed-graphs-amalgamation});
	\item finite total orders (Example~\ref{ex:amalgamation-total-orders});
	\item finite partial orders (Exercise~\ref{ex:amalgamation-partial-orders});
	\item finite trees (Section~\ref{sec:trees});
	\item finite vector spaces over the two element field (Section~\ref{sec:bit-vectors}).
\end{enumerate}
In particular, for each of the classes above, the \fraisse limit can be used as atoms, leading to algorithms for problems such as graph reachability, automaton emptiness, or automaton minimisation.