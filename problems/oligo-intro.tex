\mikexercise{\label{ex:integers-fail-orbit-finiteness} Show that the structure  $(\Int,<)$ is not oligomorphic. }
{
The power $\atoms^1$ has one orbit, but the square $\atoms^2$ has infinitely many orbits.
}



\mikexercise{For the atoms $(\mathbb Q,<)$, find all equivariant binary relations on $\atoms$. }{
All of the four equivariant relations mentioned in the solution to Exercise~\ref{ex:equivariant-binary-relations-on-equality-atoms} are still valid. (In general, when the 
		atoms gain structure, there are more equivariant sets.) However, there are four new binary relations, which refer to the total order, namely:
		\begin{align*}
			\set{(\atoma,\atomb) : \atoma < \atomb} \qquad \set{(\atoma,\atomb) : \atoma \le \atomb} \qquad \set{(\atoma,\atomb) : \atoma > \atomb} \qquad \set{(\atoma,\atomb) : \atoma \ge \atomb}.
		\end{align*}
		Observe again these are exactly the binary relations that can be defined by quantifier-free formulas. 
}

\mikexercise{Consider a structure $\atoms$ that is oligomorphic. Let $\mathbb B$ be a new structure, whose universe is a pof set over $\atoms$, and whose relations are equivariant (under automorphism of $\atoms$). Show that $\mathbb B$ is also an oligomorphic structure.  }{}


\mikexercise{\label{ex:bit-vector-orbit-count} Consider the bit vector atoms. Show that the number of orbits in $\atoms^d$ is singly exponential in $d$.}{ Here is a formula for an upper bound: 
\begin{align*}
\myunderbrace{\sum_{0 \in \set{1,\ldots,b}}}{size of a basis}
\quad 
\myunderbrace{{d \choose b}}{choose \\ the \\  basis}
\ \cdot \ 
\myunderbrace{(2^{(d-b)})^b}{choose a \\ basis\\  decomposition \\ for the \\ remaining \\ $d-b$\\ coordinates}
\qquad \leq \qquad 
d \cdot 2^d \cdot 2^{d^2}.
\end{align*}
}

\mikexercise{Consider the bit vector atoms. Find all equivariant functions of type $\atoms^2 \to \atoms$. Are these all linear maps? }{}

% \mikexercise{Consider the atoms $(\mathbb Q, <)$. Show that there is no equivariant well-founded total order on $\atoms$. }{ Suppose that $R$ is an equivariant binary relation on the atoms. Let $c$ be the smallest atom in the finite support. If $a_1<b_1$ and $a_2<b_2$ are atoms which are smaller than $c$, then $R$ selects the pair $(a_1,b_1)$ if and only if it selects the pair $(a_2,b_2)$, because these pairs can be mapped to each other by an automorphism of the rational numbers that fixes all rational numbers greater or equal to $c$. It follows that for atoms smaller than $c$, the order imposed by $R$ is either that of the rational numbers or its opposite, neither of which is well-founded.

% This example goes back to Andrzej Mostowski, who was one of the main figures in sets with atoms, which is why they are sometimes called Fraenkel-Mostowski sets. The example shows that in sets with atoms there exist sets which can be totally ordered, but not in a well-founded way. }