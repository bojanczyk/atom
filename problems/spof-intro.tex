\mikexercise{Define the  \emph{orbit count} of a deterministic pof automaton to be the number of orbits of reachable states. For a language, we can consider the set of  deterministic pof automata that recognise it, and have minimal orbit count for this property. Show that this set can contain automata that are non-isomorphic. Here, an isomorphism between two automata is an equivariant bijection 
\begin{align*}
\text{reachable states of $\Aa$}
\quad \stackrel f \longrightarrow \quad 
\text{reachable states of $\Bb$}
\end{align*}
such that for every input word, applying $f$ to the state of $\Aa$ after reading a word gives the state of $\Bb$ after reading the same word. 
} 
{
    Consider the language of one-letter words, over the alphabet $\atoms$. One automaton does not store any atoms, i.e.~it has three states ``initial'', ``one letter'', and ``at least two letters''. This corresponds to disjoint union of three copies of $\atoms^0$. 
    An alternative automaton stores the last letter in its state. Therefore, the two automata have state spaces: 
    \begin{align}
        \myunderbrace{\atoms^0 + \atoms^0 + \atoms^0}{first automaton}
        \quad \text{and} \quad
        \myunderbrace{\atoms^0 + \atoms^1 + \atoms^0}{second automaton}.
    \end{align}
    In both cases, the state space has three orbits. Clearly the first automaton is better, but this is not reflected in the orbit count alone. One can also show that an automaton with fewer orbits cannot do the trick. Indeed, the initial state must be its own orbit $\atoms^0$. This state must be rejecting, since the empty word is not in the language. The other orbit must be accepting (since otherwise the automaton would reject all words). In order to reject a word of length two or more, the automaton would need to return to the initial state, which would mean that it would accept some even longer words.
}
